%%%%%%%%%%%%%%%%%%%%%%%%%%%%%%%%%%%%%%%%%%%%

% formato FRONTE RETRO
\documentclass[epsfig,a4paper,11pt,titlepage,twoside,openany]{book}
\usepackage{epsfig}
\usepackage{plain}
\usepackage{setspace}
\usepackage[paperheight=29.7cm,paperwidth=21cm,outer=1.5cm,inner=2.5cm,top=2cm,bottom=2cm]{geometry} % per definizione layout
\usepackage{titlesec} % per formato custom dei titoli dei capitoli

\singlespacing

\usepackage[italian]{babel}

%%%%%%%%%%%%%%%%% STEFANO ADDED %%%%%%%%%%%%%%%%
% supporto lettere accentate
% queste due righe nel preambolo servono a poter utilizzare le lettere accentate in tutto il testo se no di norma si inserirebbero con \'e...
\usepackage[T1]{fontenc}
\usepackage[utf8]{inputenc}
\usepackage{hyperref} %Serve per i riferimenti
\usepackage{caption}  %Serve per le note (caption)
\usepackage{multicol}	 %Serve per usare più colonne

%%%%%%%%%%%%%%%%% ENRICO ADDED %%%%%%%%%%%%%%%%%
\usepackage{ulem} %Serve per il sottolineato
\usepackage{amsmath} %Serve per alcuni ambienti matematici
\usepackage{array} %Serve per le tabelle
\usepackage{multirow} %Seve per tabelle
\setcounter{secnumdepth}{5} %Utilissimo serve per aumentare il numero di paragrafi, si arriva fino a 5 livelli di profondità x.x.x.x.x
%GRAFICI
\usepackage{pgfplots}
\usepackage{pgfmath}
\usepackage{tikz}
%CODICE
% \usepackage{listings}
% \usepackage[cache=false]{minted} 
%\setminted{tabsize=4, breaklines, breakanywhere, linenos, mathescape,}

% \lstset{
% 	  breakatwhitespace=false,         
% 	  breaklines=true,     
% 	  basicstyle=\footnotesize\ttfamily,            
% 	  commentstyle=\color{blue}, %Indica il colore dei commenti
% 	  keywordstyle=\color{red}, %Indica il colore delle parole chiave
% 	  language=C, %Indica il linguaggio predefinito da usare
% 	  rulecolor=\color{black}, %Indica il colore dei numeri di righe
% 	  tabsize=4,
% 	  escapeinside={\%*}{*)},
% 	  morekeywords={}, %Altre parole da inserire tra le keywords. Ad esempio possiamo aggiungere do, gotttto, ecc ecc 
% }
%%%%%%%%%%%%%%%%% END OF ENRICO  %%%%%%%%%%%%%%%%

\begin{document}
%set the language of the text to italian
% !TeX spellcheck = it_IT

%%%%%%% personal commands (ALIAS):
% \newcommand{\nome_commando}[argomenti]{comando}
\newcommand{\e}[1]{$\cdot 10^{#1}$}
\newcommand{\mmax}[0]{mod\_withMax }
\newcommand{\mover}[0]{mod\_overlap }
\newcommand{\mmod}[0]{modularità modificata }
\newcommand{\nv}[0]{Node2Vec }
\newcommand{\wv}[0]{Word2Vec }
\newcommand{\cnrl}[0]{CNRL }
\newcommand{\cora}[0]{Cora }
\newcommand{\citeseer}[0]{Citeseer }
\newcommand{\LPred}[0]{Link Prediction }
%



%
%
\chapter{Sommario}\label{chap:0}
%% cos'è?
%% Il sommario è un breve riassunto del lavoro svolto dove si descrive: l’obiettivo, l’oggetto della tesi, le metodologie e le tecniche usate, i dati elaborati e la spiegazione delle conclusioni alle quali siete arrivati.
%% Il sommario dell’elaborato consiste al massimo di 3 pagine e deve contenere le seguenti informazioni: contesto e motivazioni breve riassunto del problema affrontato tecniche utilizzate e/o sviluppate risultati raggiunti, sottolineando il contributo personale del laureando/a se vi sono più persone che hanno contribuito...
Questa tesi va a ripercorrere il lavoro eseguito da \textbf{Stefano Leonardi} durante il tirocinio svolto all'interno dell'università, con la supervisione del \textbf{dottorando Nasrullah Sheikh}, per il \textbf{professore Alberto Montresor}.
%
\section{Introduzione}
Il tirocinio è basato sul documento intitolato "Community-enhanced Network Representation Learning for Network Analysis"\cite{CNRL_paper} abbreviato dalla sigla \textbf{\cnrl}. Questo testo propone un innovativo metodo per l'individuazione delle comunità all'interno dei grafi. Lo scopo di questo lavoro è quello di replicare i risultati ivi proposti. Una volta compreso il codice che ne è la base, abbiamo cercato di migliorare le prestazioni dell'algoritmo, anche sfruttando elementi non originariamente considerati.\newline
L'algoritmo proposto nell'articolo utilizza diverse tecniche, alcune delle quali dettagliatamente spiegate in altri documenti:
\begin{itemize}
	\item "DeepWalk: Online Learning of Social Representations"\cite{DW_paper}
	\item "node2vec: Scalable Feature Learning for Networks"\cite{N2V_paper}
	\item "How exactly does word2vec work?"\cite{W2V_paper}
\end{itemize} 
Segue la spiegazione del perché l'individuazione di comunità è una tecnica così importante.
%
\section{Cosa sono i grafi e cosa le comunità}
La base sono i grafi, una particolare struttura dati composta da \textbf{nodi/vertici} $V$ (dove $n=|V|$) e \textbf{archi/bordi/lati} $E$ (dove $m=|E|$). I primi possono rappresentare un'entità eventualmente anche con l'aiuto di attributi. Diversamente gli archi sono dei collegamenti che legano due nodi. Possono essere interamente \textbf{orientati/diretti} o \textbf{non orientati/indiretti}. Questa struttura dati permette di rappresentare molte situazioni e per tale motivo è importante.\\
Lo scopo dell'individuazione delle comunità è riconoscere dei gruppi di nodi simili fra loro.\\
Si fa notare che i nodi risultano essere simili per diverse ragioni. Possono avere le stesse caratteristiche strutturali in quanto sono dei cardini che legano molti altri elementi, oppure punti isolati da tutto il resto, o più semplicemente  valori analoghi sugli attributi.\\
Gruppi di nodi simili possono essere gestiti in maniera omogenea; questo permette di sfruttarne le peculiarità. La versatilità d'applicazione dei grafi conferisce a questi algoritmi una grande importanza.
%
\section{Come si valutano le comunità}
Date due partizioni differenti su un grafo, è necessario capire quale di queste è la migliore. Se il grafo è molto piccolo, tanto da poterlo comprendere guardando unicamente una sua rappresentazione grafica, e le comunità individuate sono poche, si potrebbe visivamente decidere anche ad occhio qual è la partizione migliore.\\
Se invece le comunità individuate sono tante e il grafo non è piccolo, allora non è facile effettuare una scelta e ancor meno riuscire a giustificarla. Esistono metodi formali per analizzare e decidere quale partizione scegliere: questi metodi prendono il nome di \textbf{modularità}.\\
\\
Le \textbf{partizioni} vengono così definite:
\begin{itemize}
	\item Una partizione è un insieme di comunità
	\item Non tutte le comunità hanno le stesse dimensioni, ossia lo stesso numero di nodi che vi appartengono
	\item Le comunità non hanno né una dimensione massima né una dimensione minima.\\
	Si può considerare che la loro dimensione cada nell'intervallo $[1, n]$. Il limite inferiore è $1$ perché devono contenere almeno un elemento, mentre il limite superiore è $n$ perché non possono contenere più nodi di quelli esistenti nel grafo, che sono in numero di $n$
	\item È possibile che un nodo non appartenga a nessuna comunità
	\item Un nodo del grafo può appartenere a più di una comunità, potenzialmente anche a tutte quelle presenti. Se questo accade significa che c'è una sovrapposizione di comunità
\end{itemize}
La metrica di valutazione di una partizione è la modularità (il cui simbolo è $Q$). Questo nome può essere associato a diverse metriche. A seconda di cosa si cerca si avrà interesse a valorizzare alcuni aspetti e penalizzarne altri. Per tale motivo la modularità è solo il nome usato per indicare il valore dato dalla valutazione di una partizione.\\
Durante il tirocinio sono stati adottati tre metodi differenti: \textbf{\mmax}, \textbf{\mover} e \textbf{\mmod}. I primi due nomi non sono ufficiali in quanto sono stati scelti al solo scopo di riconoscerli, mentre l'ultimo, \mmod, è quello utilizzato nell'articolo su \cnrl \cite{CNRL_paper}.\\
I dettagli sul funzionamento e sulla motivazione del perché sono state scelte sono all'interno dell'apposito capitolo sulle metriche di valutazione (\autoref{chap:2}).
%
\section{I cambiamenti apportati}
Per calcolare una partizione, il codice di \cnrl\ utilizza un sistema di visite random (casuali) e con questo ricostruisce le possibili somiglianze fra i nodi. Non volendo toccare questa sezione, le modifiche da noi apportate si inseriscono all'interno della creazione dei cammini.\\
Un cammino è definito come una sequenza di nodi. Preso un elemento qualsiasi fra questi, deve essere possibile arrivare all'elemento successivo attraversando un solo arco. I due elementi devono essere direttamente connessi, almeno nel senso di percorrenza.\\
L'algoritmo genera $x$ cammini, ognuno di lunghezza massima $l$, sul grafo, dove $x=n \cdot w$:
\begin{itemize}
	\item $n$: è il numero di nodi del grafo
	\item $w$: è il numero di cammini che si fanno partire da ogni singolo nodo
	\item $l$: è la lunghezza massima di ogni cammino
\end{itemize}
%
Con questi dati, l'algoritmo di \cnrl\ calcola la suddivisione in comunità. Si può notare come tutti questi dati vengano estratti esclusivamente dalla struttura del grafo, ossia da come nodi ed archi sono interconnessi.\\
L'intuizione sfruttata per modificare l'algoritmo si basa sull'esistenza di altri dati non utilizzati, ossia gli attributi dei nodi. Tramite la creazione di nuovi particolari cammini è possibile far risultare vicini due nodi in precedenza lontani, in quanto non direttamente connessi. Risultano vicini perché questi nodi condividono uno o più attributi. L'introduzione dei nuovi dati porta ad un radicale cambiamento della suddivisione in comunità.\\
L'idea di introdurre gli attributi per migliorare l'efficienza dell'algoritmo è sorta grazie agli articoli:
\begin{itemize}
	\item "Attributed social network embedding"\cite{SNE_paper}
	\item "Network Representation Learning with Rich Text Information"\cite{TADW_paper}
\end{itemize}
%
\section{Esperimenti}
Gli esperimenti che sono stati eseguiti nel corso del tirocinio si dividono in tre principali gruppi. Ognuno di questi è dedicato ad un diverso algoritmo.\\
Le tre aree sono:
\begin{itemize}
	\item \textbf{\LPred}: Mediante la creazione di archi verosimili e potenzialmente reali, ma che non appaiono in una particolare selezione denominata "test", va a verificare se sono più coerenti gli archi generati casualmente rispetto a quelli reali.\\
	Per assegnare un valore comparabile ad un arco, si guarda la relazione, che lega i due nodi agli estremi del collegamento, valutata mediante apposite funzioni di similarità.\\
	Gli archi dei due insiemi sono in seguito accoppiati e, sulla base del valore più alto, si assegna un punteggio. Calcolando il punteggio per ogni coppia si andrà a rappresentare l'attendibilità per predire in maniera corretta un nuovo arco.
	\item \textbf{Classificazione dei nodi} è una tecnica che necessita di una rappresentazione univoca dei vertici. Ognuno è definito come un punto all'interno di uno spazio a $64$ dimensioni e se ne conosce la classe d'appartenenza. Questo permette di predire per un qualsiasi nuovo nodo rappresentato nella stessa forma degli altri, a quale classe andrà ad appartenere.\\
	I test svolti analizzano con quale percentuale il classificatore, ossia la funzione che assegna la classe, riesce a predire in maniera esatta la classe di un nodo.
	\item La \textbf{valutazione dell'individuazione di comunità} si occupa in prima istanza di creare le comunità di nodi simili fra loro, secondo determinate caratteristiche. I metodi della modularità valutano se le partizioni individuate sono ben costruite oppure no.
\end{itemize}
%
\section{Com'è strutturata la tesi}
Il \autoref{chap:1} sull'implementazione spiega le dinamiche base dell'algoritmo principale, da cui è partito tutto il lavoro. Poi viene descritta l'idea per un incremento di prestazioni mediante gli attributi, tramite un esempio di come viene applicata.\\
Nel \autoref{chap:2} sono esposte le metriche di valutazione fulcro degli esperimenti principali. Per ognuna son spiegate le dinamiche e le motivazioni per cui son state scelte.\\
Il \autoref{chap:3} è dedicato agli esperimenti. Vengono fornite le conoscenze di base per comprendere i tre algoritmi su cui s'è lavorato. Per ognuno di questi vengono descritte le meccaniche, tramite un esempio e poi sono illustrati i risultati raggiunti.\\
Si termina con le conclusioni (\autoref{chap:4}) e la bibliografia.
%

\newpage
%\end{document}



