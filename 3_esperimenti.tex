%%%%%%%%%%%%%%%%%%%%%%%%%%%%%%%%%%%%%%%%%%%

% formato FRONTE RETRO
\documentclass[epsfig,a4paper,11pt,titlepage,twoside,openany]{book}
\usepackage{epsfig}
\usepackage{plain}
\usepackage{setspace}
\usepackage[paperheight=29.7cm,paperwidth=21cm,outer=1.5cm,inner=2.5cm,top=2cm,bottom=2cm]{geometry} % per definizione layout
\usepackage{titlesec} % per formato custom dei titoli dei capitoli

\singlespacing

\usepackage[italian]{babel}

%%%%%%%%%%%%%%%%% STEFANO ADDED %%%%%%%%%%%%%%%%
% supporto lettere accentate
% queste due righe nel preambolo servono a poter utilizzare le lettere accentate in tutto il testo se no di norma si inserirebbero con \'e...
\usepackage[T1]{fontenc}
\usepackage[utf8]{inputenc}
\usepackage{hyperref} %Serve per i riferimenti
\usepackage{caption}  %Serve per le note (caption)
\usepackage{multicol}	 %Serve per usare più colonne

%%%%%%%%%%%%%%%%% ENRICO ADDED %%%%%%%%%%%%%%%%%
\usepackage{ulem} %Serve per il sottolineato
\usepackage{amsmath} %Serve per alcuni ambienti matematici
\usepackage{array} %Serve per le tabelle
\usepackage{multirow} %Seve per tabelle
\setcounter{secnumdepth}{5} %Utilissimo serve per aumentare il numero di paragrafi, si arriva fino a 5 livelli di profondità x.x.x.x.x
%GRAFICI
\usepackage{pgfplots}
\usepackage{pgfmath}
\usepackage{tikz}
%CODICE
% \usepackage{listings}
% \usepackage[cache=false]{minted} 
%\setminted{tabsize=4, breaklines, breakanywhere, linenos, mathescape,}

% \lstset{
% 	  breakatwhitespace=false,         
% 	  breaklines=true,     
% 	  basicstyle=\footnotesize\ttfamily,            
% 	  commentstyle=\color{blue}, %Indica il colore dei commenti
% 	  keywordstyle=\color{red}, %Indica il colore delle parole chiave
% 	  language=C, %Indica il linguaggio predefinito da usare
% 	  rulecolor=\color{black}, %Indica il colore dei numeri di righe
% 	  tabsize=4,
% 	  escapeinside={\%*}{*)},
% 	  morekeywords={}, %Altre parole da inserire tra le keywords. Ad esempio possiamo aggiungere do, gotttto, ecc ecc 
% }
%%%%%%%%%%%%%%%%% END OF ENRICO  %%%%%%%%%%%%%%%%

\begin{document}
%set the language of the text to italian
% !TeX spellcheck = it_IT

%%%%%%% personal commands (ALIAS):
% \newcommand{\nome_commando}[argomenti]{comando}
\newcommand{\e}[1]{$\cdot 10^{#1}$}
\newcommand{\mmax}[0]{mod\_withMax }
\newcommand{\mover}[0]{mod\_overlap }
\newcommand{\mmod}[0]{modularità modificata }
\newcommand{\nv}[0]{Node2Vec }
\newcommand{\wv}[0]{Word2Vec }
\newcommand{\cnrl}[0]{CNRL }
\newcommand{\cora}[0]{Cora }
\newcommand{\citeseer}[0]{Citeseer }
\newcommand{\LPred}[0]{Link Prediction }
%



%
\chapter{Esperimenti}
\section{Origini dei grafi}
In questa sezione vengono spiegati i dataset utilizzati e le loro origini.\\
Si possono dividere in due gruppi, non solo perché tali gruppi al loro interno presentino caratteristiche d'affinità, ma anche in quanto i grafi che vi appartengono sono stati utilizzati nel corso del tirocinio per due applicazioni differenti. Nella Tabella~\ref{tab:dati_grafi} sono elencati i dettagli tecnici di ogni dataset.
%
\begin{center}
	\begin{tabular}{|l|r|r|c|c|r|}
		\hline
		grafi&nodi&archi&diretto&etichette&attributi\\
		\hline
		Cora & 2708 & 5429 & sì & 7 & 1433\\
		Citeseer & 3312 & 4732 & sì & 6 & 3703\\
		\hline
		BlogCatalog & 10312 & 333983 & no & 39 & 0\\
		Gnutella & 6301 & 20777 & no & 0 & 0\\
		Dolphins & 62 & 159 & no & 0 & 0\\
		Karate & 34 & 78 & no & 0 & 0\\
		\hline
		\end{tabular}
		\captionof{table}{Per ogni grafo sono indicate le sue caratteristiche principali}
		\label{tab:dati_grafi}
\end{center}
%
\subsection*{Esperimento principale: Cora - Citeseer}\cite{Co-Ci_1}\cite{Co-Ci_2}
Il dataset di \textbf{Cora} è un grafo diretto che rappresenta una rete di articoli scientifici sull'apprendimento automatico, ogni nodo è un articolo. Mentre ogni arco rappresenta il collegamento fra un documento è l'altro, se il documento $A$ cita il documento $B$ si avrà l'arco $(A, B)$. La rete è costruita in modo che ogni nodo abbia almeno un arco entrante o uscente.\\
Ogni articolo appartiene ad esattamente una classe identificata da un etichetta. Ogni attributo rappresenta la presenza (deve apparire almeno 10 volte) o meno di una determinata parola nel testo del documento.\\
\\
Il dataset di \textbf{Citeseer} è un grafo diretto che rappresenta una rete analoga a quella di Cora, anche qui si hanno articoli scientifici che si citano vicendevolmente. L'etichetta è la classe d'appartenenza cui l'articolo appartiene, e gli attributi son nuovamente la presenza o meno di certe parole nel testo.\\
%
\subsection*{Grafi per esperimenti minori}
Questi quattro grafi vengono presentati in ordine decrescente nel numero dei nodi, così come fatto nella seconda parte della Tabella~\ref{tab:dati_grafi}.\\
\begin{itemize}
	\item \textbf{BlogCatalog}\cite{BlogCatalog} rappresenta la rete di relazioni, di conoscenza, fra gli utenti di alcune piattaforme per blog
	\item \textbf{Gnutella}\cite{Gnutella_1}\cite{Gnutella_2} questo grafo si rifà alla rete peer-to-peer per lo scambio di file di Gnutella nell'agosto 2002. Ogni nodo rappresenta un host e ogni arco un collegamento fra due host
	\item \textbf{Soc-Dolphins}\cite{Dolphins_1}\cite{Dolphins_2} riporta la rete sociale di un gruppo di delfini al largo della Nuova Zelanda nel 2003
	\item \textbf{Karate}\cite{Karate} raccoglie lo storico degli incontri in un club di karate universitario nel 1977. Ogni nodo è un membro del club, ogni arco indica uno scontro terminato in pareggio
\end{itemize}
%
%\section{Valutazione dell'individuazione di comunità}
%
%\section{Classificazione di nodi}
%



%
\end{document}